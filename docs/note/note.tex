\documentclass[12pt, onecolumn]{article}
\setlength\paperwidth{23cm}
% 引入相关的包
\usepackage{amsmath, listings, fontspec, geometry, graphicx, ctex, color, subfigure, amsfonts, amssymb}
\usepackage{multirow}
\usepackage[table,xcdraw]{xcolor}
\usepackage[ruled]{algorithm2e}
\usepackage[hidelinks]{hyperref}
\hypersetup{
	colorlinks=true,
	linkcolor=red,
	citecolor=red,
}

% 设定页面的尺寸和比例
\geometry{left = 1.5cm, right = 1.5cm, top = 1.5cm, bottom = 1.5cm}

% 设定两栏之间的间距
\setlength\columnsep{1cm}

% 设定字体,为代码的插入作准备
\newfontfamily\ubuntu{Ubuntu Mono}
\newfontfamily\consolas{Consolas}

% 头部信息
\title{\normf{RIs时空标定方法}}
\author{\normf{陈烁龙}}
\date{\normf{\today}}

% 代码块的风格设定
\lstset{
	language=C++,
	basicstyle=\scriptsize\ubuntu,
	keywordstyle=\textbf,
	stringstyle=\itshape,
	commentstyle=\itshape,
	numberstyle=\scriptsize\ubuntu,
	showstringspaces=false,
	numbers=left,
	numbersep=8pt,
	tabsize=2,
	frame=single,
	framerule=1pt,
	columns=fullflexible,
	breaklines,
	frame=shadowbox, 
	backgroundcolor=\color[rgb]{0.97,0.97,0.97}
}

% 字体族的定义
% \fangsong \songti \heiti \kaishu
\newcommand\normf{\fangsong}
\newcommand\boldf{\heiti}
\newcommand\keywords[1]{\boldf{关键词:} \normf #1}

\newcommand\liehat[1]{\left[ #1 \right]_\times}
\newcommand\lievee[1]{\left[ #1 \right]^\vee}
\newcommand\liehatvee[1]{\left[ #1 \right]^\vee_\times}

\newcommand\mynote[1]{{\bf{\emph{\textcolor{blue}{* \normf{#1} ...}}}}}

\newcommand\mlcomment[1]{\iffalse #1 \fi}
%\newcommand\mlcomment[1]{ #1 }

\begin{document}
	
	% 插入头部信息
	\maketitle
	% 换页
	\thispagestyle{empty}
	\clearpage
	
	% 插入目录、图、表并换页
	\pagenumbering{roman}
	\tableofcontents
	\newpage
	\listoffigures
	\newpage
	\listoftables
	% 罗马字母形式的页码
	
	\clearpage
	% 从该页开始计数
	\setcounter{page}{1}
	% 阿拉伯数字形式的页码
	\pagenumbering{arabic}
	
	
	\section{\normf{Radar Static Measurement (V1)}}
	The continuous-time trajectory is the one of other sensor (e.g., IMU), we have:
	\begin{equation}
	{^{b_0}\boldsymbol{p}_t}={^{b_0}_{b}\boldsymbol{R}(\tau)}\cdot{^{b}\boldsymbol{p}_t(\tau)}+{^{b_0}\boldsymbol{p}_{b}(\tau)}
	\end{equation}
	with
	\begin{equation}
	{^{b}\boldsymbol{p}_t(\tau)}={^{b}_{r}\boldsymbol{R}}\cdot{^{r}\boldsymbol{p}_t(\tau)}+{^{b}\boldsymbol{p}_{r}}
	\end{equation}
	then
	\begin{equation}
	{^{b_0}\boldsymbol{p}_t}={^{b_0}_{b}\boldsymbol{R}(\tau)}\cdot{^{b}_{r}\boldsymbol{R}}\cdot{^{r}\boldsymbol{p}_t(\tau)}
	+{^{b_0}_{b}\boldsymbol{R}(\tau)}\cdot{^{b}\boldsymbol{p}_{r}}
	+{^{b_0}\boldsymbol{p}_{b}(\tau)}
	\end{equation}
	differentiate
	\begin{equation}
	{^{b_0}\dot{\boldsymbol{p}}_t}=\boldsymbol{0}_{3\times 1}=
	-\liehat{{^{b_0}_{b}\boldsymbol{R}(\tau)}\cdot{^{b}_{r}\boldsymbol{R}}\cdot{^{r}\boldsymbol{p}_t(\tau)}}\cdot{^{b_0}_{b}\dot{\boldsymbol{R}}(\tau)}
	+{^{b_0}_{b}\boldsymbol{R}(\tau)}\cdot{^{b}_{r}\boldsymbol{R}}\cdot{^{r}\dot{\boldsymbol{p}}_t(\tau)}
	-\liehat{{^{b_0}_{b}\boldsymbol{R}(\tau)}\cdot{^{b}\boldsymbol{p}_{r}}}\cdot{^{b_0}_{b}\dot{\boldsymbol{R}}(\tau)}
	+{^{b_0}\dot{\boldsymbol{p}}_{b}(\tau)}
	\end{equation}
	\begin{equation}
	\boldsymbol{0}_{3\times 1}=
	-\liehat{{^{b}_{r}\boldsymbol{R}}\cdot{^{r}\boldsymbol{p}_t(\tau)}}\cdot{^{b_0}_{b}\boldsymbol{R}^\top(\tau)}\cdot{^{b_0}_{b}\dot{\boldsymbol{R}}(\tau)}
	+{^{b}_{r}\boldsymbol{R}}\cdot{^{r}\dot{\boldsymbol{p}}_t(\tau)}
	-\liehat{{^{b}\boldsymbol{p}_{r}}}\cdot{^{b_0}_{b}\boldsymbol{R}^\top(\tau)}\cdot{^{b_0}_{b}\dot{\boldsymbol{R}}(\tau)}
	+{^{b_0}_{b}\boldsymbol{R}^\top(\tau)}\cdot{^{b_0}\dot{\boldsymbol{p}}_{b}(\tau)}
	\end{equation}
	\begin{equation}
	\boldsymbol{0}_{3\times 1}=
	-\liehat{{^{r}\boldsymbol{p}_t(\tau)}}\cdot{^{b}_{r}\boldsymbol{R}^\top}\cdot{^{b_0}_{b}\boldsymbol{R}^\top(\tau)}\cdot{^{b_0}_{b}\dot{\boldsymbol{R}}(\tau)}
	+{^{r}\dot{\boldsymbol{p}}_t(\tau)}
	-{^{b}_{r}\boldsymbol{R}^\top}\cdot\liehat{{^{b}\boldsymbol{p}_{r}}}\cdot{^{b_0}_{b}\boldsymbol{R}^\top(\tau)}\cdot{^{b_0}_{b}\dot{\boldsymbol{R}}(\tau)}
	+{^{b}_{r}\boldsymbol{R}^\top}\cdot{^{b_0}_{b}\boldsymbol{R}^\top(\tau)}\cdot{^{b_0}\dot{\boldsymbol{p}}_{b}(\tau)}
	\end{equation}
	thus, the velocity of target $\{t\}$ with respect to the radar $\{r\}$ parameterized in $\{r\}$ could be expressed as:
	\begin{equation}
	{^{r}\dot{\boldsymbol{p}}_t(\tau)}=
	\liehat{{^{r}\boldsymbol{p}_t(\tau)}}\cdot{^{b}_{r}\boldsymbol{R}^\top}\cdot{^{b_0}_{b}\boldsymbol{R}^\top(\tau)}\cdot{^{b_0}_{b}\dot{\boldsymbol{R}}(\tau)}
	+{^{b}_{r}\boldsymbol{R}^\top}\cdot\liehat{{^{b}\boldsymbol{p}_{r}}}\cdot{^{b_0}_{b}\boldsymbol{R}^\top(\tau)}\cdot{^{b_0}_{b}\dot{\boldsymbol{R}}(\tau)}
	-{^{b}_{r}\boldsymbol{R}^\top}\cdot{^{b_0}_{b}\boldsymbol{R}^\top(\tau)}\cdot{^{b_0}\dot{\boldsymbol{p}}_{b}(\tau)}
	\end{equation}
	
	\section{\normf{Radar Static Measurement (V2)}}
	\begin{equation}
	{^{b_0}{\boldsymbol{p}}_r(\tau)}={^{b_0}_{b}\boldsymbol{R}(\tau)}\cdot{^{b}{\boldsymbol{p}}_r}+{^{b_0}{\boldsymbol{p}}_b(\tau)}
	\end{equation}
	\begin{equation}
	{^{b_0}\dot{\boldsymbol{p}}_r(\tau)}=-\liehat{{^{b_0}_{b}\boldsymbol{R}(\tau)}\cdot{^{b}{\boldsymbol{p}}_r}}\cdot{^{b_0}_{b}\dot{\boldsymbol{R}}(\tau)}
	+{^{b_0}\dot{\boldsymbol{p}}_b(\tau)}
	\end{equation}
	thus, the velocity of radar $\{r\}$ with respect to the frame $\{b_0\}$ parameterized in $\{r\}$ could be expressed as:
	\begin{equation}
	{^{b}_{r}\boldsymbol{R}^\top}\cdot{^{b_0}_{b}\boldsymbol{R}^\top(\tau)}\cdot{^{b_0}\dot{\boldsymbol{p}}_r(\tau)}=
	{^{b}_{r}\boldsymbol{R}^\top}\cdot{^{b_0}_{b}\boldsymbol{R}^\top(\tau)}\cdot
	\left(
	-\liehat{{^{b_0}_{b}\boldsymbol{R}(\tau)}\cdot{^{b}{\boldsymbol{p}}_r}}\cdot{^{b_0}_{b}\dot{\boldsymbol{R}}(\tau)}
	+{^{b_0}\dot{\boldsymbol{p}}_b(\tau)}
	\right) 
	\end{equation}
	actually, by introducing ${^{r}\boldsymbol{p}_t^\top(\tau)}$, we have:
	\begin{equation}
	{^{r}\boldsymbol{p}_t^\top(\tau)}\cdot{^{b}_{r}\boldsymbol{R}^\top}\cdot{^{b_0}_{b}\boldsymbol{R}^\top(\tau)}\cdot{^{b_0}\dot{\boldsymbol{p}}_r(\tau)}=
	-{^{r}\boldsymbol{p}_t^\top(\tau)}\cdot{^{r}\dot{\boldsymbol{p}}_t(\tau)}
	\end{equation}
	
	\section{\normf{样条恢复(Rotation \& Velocity B-splines)和重力估计}}
	\normf
	对于旋转样条,使用IMU陀螺仪量测进行拟合。对于Radar量测,有:
	\begin{equation}
	r\cdot v_r=
	{^{r}\boldsymbol{p}_t^\top(\tau)}\cdot{^{r}\dot{\boldsymbol{p}}_t(\tau)}
	\end{equation}
	或者:
	\begin{equation}
	r\cdot v_r=-
	{^{r}\boldsymbol{p}_t^\top(\tau)}\cdot{^{b}_{r}\boldsymbol{R}^\top}\cdot{^{b_0}_{b}\boldsymbol{R}^\top}\cdot{^{b_0}\dot{\boldsymbol{p}}_r(\tau)}
	\end{equation}
	
	对于动态target,有:
	\begin{equation}
	r\cdot v_r=
	{^{r}\boldsymbol{p}_t^\top(\tau)}\cdot
	\left( 
	{^{r}\dot{\boldsymbol{p}}_t(\tau)}-
	{^{b}_{r}\boldsymbol{R}^\top}\cdot{^{b_0}_{b}\boldsymbol{R}^\top(\tau)}\cdot
	{^{b_0}\dot{\boldsymbol{p}}_t}
	\right) 
	\end{equation}
	因为假设了target为静态,所以当有动态target纳入进来时,需要loss函数,一般考虑的最大容许外点为:
	\begin{equation}
	\frac{{^{r}\boldsymbol{p}_t^\top(\tau)}\cdot{^{b}_{r}\boldsymbol{R}^\top}\cdot{^{b_0}_{b}\boldsymbol{R}^\top(\tau)}\cdot
		{^{b_0}\dot{\boldsymbol{p}}_t}}{r}
	\eqsim
	\frac{{^{r}\boldsymbol{p}_t^\top(\tau)}\cdot
		{^{b_0}\dot{\boldsymbol{p}}_t}}{r}
	\eqsim
	\cos{45^\circ}\cdot
	\left\| {^{b_0}\dot{\boldsymbol{p}}_t}\right\| 
	\le
	\frac{\sqrt 2}{2}\cdot \frac{1}{2}\;(m/s^2)
	=\frac{\sqrt 2}{4}\;(m/s^2)
	\xrightarrow{(\cdot)^2}0.125
	\end{equation}

	对于IMU量测, 基于连续时间的预积分(速度层面),有:
	\begin{equation}
	{^{b}\boldsymbol{a}(t)}={^{b_0}_{b}\boldsymbol{R}^\top(t)}\cdot\left( {^{b_0}\ddot{\boldsymbol{p}}_b(t)}-{^{b_0}\boldsymbol{g}}\right)
	\quad\to\quad
	{^{b_0}_{b}\boldsymbol{R}(t)}\cdot{^{b}\boldsymbol{a}(t)}= {^{b_0}\ddot{\boldsymbol{p}}_b(t)}-{^{b_0}\boldsymbol{g}}
	\end{equation}
	在$\tau\in[t_k,t_{k+1}]$时间段内,有:
	\begin{equation}
	\int_{\tau\in[t_k,t_{k+1}]}{^{b_0}_{b}\boldsymbol{R}(\tau)}\cdot{^{b}\boldsymbol{a}(\tau)}\cdot d\tau=
	{^{b_0}\dot{\boldsymbol{p}}_b(t_{k+1})}-{^{b_0}\dot{\boldsymbol{p}}_b(t_k)}
	-{^{b_0}\boldsymbol{g}}\cdot (t_{k+1}-t_k)
	\end{equation}
	
	\section{\normf{分步初始化}}
	首先拟合SO3样条,然后初始化外参,再恢复速度曲线。
	
	首先通过最小二乘,计算得到${^{b}_{r}\boldsymbol{R}^\top}\cdot{^{b_0}_{b}\boldsymbol{R}^\top}\cdot{^{b_0}\dot{\boldsymbol{p}}_r(\tau)}$,记为${^{b_0}\dot{\boldsymbol{p}}^r_r(\tau)}$:
	\begin{equation}
	r\cdot v_r=-
	{^{r}\boldsymbol{p}_t^\top(\tau)}\cdot{^{b}_{r}\boldsymbol{R}^\top}\cdot{^{b_0}_{b}\boldsymbol{R}^\top}\cdot{^{b_0}\dot{\boldsymbol{p}}_r(\tau)}
	=-{^{r}\boldsymbol{p}_t^\top(\tau)}\cdot{^{b_0}\dot{\boldsymbol{p}}^r_r(\tau)}
	\end{equation}
	然后,我们有:
	\begin{equation}
	{^{b_0}\dot{\boldsymbol{p}}_b(\tau)}={^{b_0}_{b}\boldsymbol{R}}(\tau)\cdot{^{b}_{r}\boldsymbol{R}}\cdot{^{b_0}\dot{\boldsymbol{p}}^r_r(\tau)}
	+\liehat{{^{b_0}_{b}\boldsymbol{R}(\tau)}\cdot{^{b}{\boldsymbol{p}}_r}}\cdot{^{b_0}_{b}\dot{\boldsymbol{R}}(\tau)}
	\end{equation}
	所以在$\tau\in[t_k,t_{k+1}]$时间段内,有::
	\begin{equation}
	\int_{\tau\in[t_k,t_{k+1}]}{^{b_0}_{b}\boldsymbol{R}(\tau)}\cdot{^{b}\boldsymbol{a}(\tau)}\cdot d\tau=
	{^{b_0}\dot{\boldsymbol{p}}_b(t_{k+1})}-{^{b_0}\dot{\boldsymbol{p}}_b(t_k)}
	-{^{b_0}\boldsymbol{g}}\cdot (t_{k+1}-t_k)
	\end{equation}
	
	\section{\normf{批处理优化}}
	在批处理优化中,一共有三种因子:
	\begin{enumerate}
		\item IMU加速度因子:涉及重力优化,加速度计内参,和速度样条和旋转样条优化;
		\begin{equation}
		{^{b}\boldsymbol{a}(t)}={^{b_0}_{b}\boldsymbol{R}^\top(t)}\cdot\left( {^{b_0}\ddot{\boldsymbol{p}}_b(t)}-{^{b_0}\boldsymbol{g}}\right)
		\end{equation}
		
		\item IMU陀螺仪因子:涉及陀螺仪内参优化,旋转样条优化;
		\begin{equation}
		{^{b}\boldsymbol{\omega}(t)}={^{b_0}_{b}\boldsymbol{R}^\top(t)}\cdot{^{b_0}_{b}\dot{\boldsymbol{R}}(t)}
		\end{equation}
		
		\item Radar因子:涉及Radar相对于IMU的外参和时延的优化,速度样条和旋转样条优化;
		
	\end{enumerate}

	\section{\normf{多IMU因子}}
	\normf
	如果是单一IMU,则其输出和位姿轨迹之间的关系为:
	\begin{equation}
	\label{equ:imu_bspline}
	\begin{cases}
	\begin{aligned}
	{^{b}\boldsymbol{a}(t)}&={^{b_0}_{b}\boldsymbol{R}^\top(t)}\cdot({^{b_0}\ddot{\boldsymbol{p}}_I(t)}-{^{b_0}\boldsymbol{g}})\\
	{^{b}\boldsymbol{\omega}(t)}&={^{b_0}_{b}\boldsymbol{R}^\top(t)}\cdot{^{b_0}_{b}\dot{\boldsymbol{R}}(t)}
	\end{aligned}
	\end{cases}
	\end{equation}
	其中$b_0$是参考坐标系(即位姿曲线表达所参考的坐标系),这里参考的是第一个IMU坐标系,该参考坐标系可以任意更换。假设现在有$n$个IMU刚性连接在一起,另有一个虚拟构建的参考IMU,标号为$c$,则对于第$i$个IMU,有:
	\begin{equation}
	\begin{cases}
	\begin{aligned}
	{^{b^i}\boldsymbol{a}(t)}&={^{b^c_0}_{b^i}\boldsymbol{R}^\top(t)}\cdot({^{b^c_0}\ddot{\boldsymbol{p}}_{b^i}(t)}-{^{b^c_0}\boldsymbol{g}})
	\\
	{^{b^i}\boldsymbol{\omega}(t)}&={^{b^c_0}_{b^i}\boldsymbol{R}^\top(t)}\cdot{^{b^c_0}_{b^i}\dot{\boldsymbol{R}}(t)}
	\end{aligned}
	\end{cases}
	\end{equation}
	将上式中${^{b^c_0}_{b^i}\boldsymbol{R}(t)}$、${^{b^c_0}_{b^i}\dot{\boldsymbol{R}}(t)}$、${^{b^c_0}\ddot{\boldsymbol{p}}_{b^i}(t)}$表达为参考IMU的位姿轨迹和外参的组合(位姿求导使用左扰动):
	\begin{equation}
	{^{b^c_0}_{b^i}\boldsymbol{T}(t)}={^{b^c_0}_{b^c}\boldsymbol{T}(t)}\cdot{^{b^c}_{b^i}\boldsymbol{T}}\to
	\begin{cases}
	\begin{aligned}
	{^{b^c_0}_{b^i}\boldsymbol{R}(t)}&={^{b^c_0}_{b^c}\boldsymbol{R}(t)}\cdot{^{b^c}_{b^i}\boldsymbol{R}}\\
	{^{b^c_0}\boldsymbol{p}_{b^i}(t)}&={^{b^c_0}_{b^c}\boldsymbol{R}(t)}\cdot{^{b^c}\boldsymbol{p}_{b^i}}+{^{b^c_0}\boldsymbol{p}_{b^c}(t)}
	\end{aligned}
	\end{cases}
	\end{equation}
	\begin{equation}
	\begin{cases}
	\begin{aligned}
	{^{b^c_0}_{b^i}\dot{\boldsymbol{R}}(t)}&={^{b^c_0}_{b^c}\dot{\boldsymbol{R}}(t)}\\
	{^{b^c_0}\dot{\boldsymbol{p}}_{b^i}(t)}&=
	-\liehat{{^{b^c_0}_{b^c}\boldsymbol{R}(t)}\cdot{^{b^c}\boldsymbol{p}_{b^i}}}\cdot{^{b^c_0}_{b^c}\dot{\boldsymbol{R}}(t)}
	+{^{b^c_0}\dot{\boldsymbol{p}}_{b^c}(t)}
	\\
	{^{b^c_0}\ddot{\boldsymbol{p}}_{b^i}(t)}&=
	-\liehat{{^{b^c_0}_{b^c}\boldsymbol{R}(t)}\cdot{^{b^c}\boldsymbol{p}_{b^i}}}\cdot
	{^{b^c_0}_{b^c}\ddot{\boldsymbol{R}}(t)}+{^{b^c_0}\ddot{\boldsymbol{p}}_{b^c}(t)}
	-\liehat{{^{b^c_0}_{b^c}\dot{\boldsymbol{R}}(t)}}\cdot
	\liehat{{^{b^c_0}_{b^c}\boldsymbol{R}(t)}\cdot{^{b^c}\boldsymbol{p}_{b^i}}}\cdot{^{b^c_0}_{b^c}\dot{\boldsymbol{R}}(t)}
	\end{aligned}
	\end{cases}
	\end{equation}
	由于是左扰动,所以角速度向量、加速度向量都表达在全局坐标系$b^c_0$下。这样,基于参考IMU轨迹和外参,可以计算出其余IMU的输出,即可以构建约束。优化的参数可以是参考IMU位姿轨迹、外参、重力向量在参考IMU第一帧坐标系下的坐标。
	
	预积分,在$\tau\in[t_k-t_{k+1}]$时间段内,有:
	\begin{equation}
	\int_{\tau\in[t_k,t_{k+1}]}{^{b_0^c}_{b^c}\boldsymbol{R}(\tau)}\cdot{^{b^c}\boldsymbol{a}(\tau)}\cdot d\tau=
	{^{b_0^c}\dot{\boldsymbol{p}}_{b^c}(t_{k+1})}-{^{b_0^c}\dot{\boldsymbol{p}}_{b^c}(t_k)}
	-{^{b_0^c}\boldsymbol{g}}\cdot (t_{k+1}-t_k)
	\end{equation}
	其中${^{b_0^c}_{b^c}\boldsymbol{R}(\tau)}\cdot{^{b^c}\boldsymbol{a}(\tau)}$可以通过下式计算得到:
	\begin{equation}
	\begin{aligned}
	{^{b^i}\boldsymbol{a}(t)}&=\left( {^{b^c_0}_{b^c}\boldsymbol{R}(t)}\cdot{^{b^c}_{b^i}\boldsymbol{R}}
	\right)^\top
	\cdot\left( {^{b_0^c}\ddot{\boldsymbol{p}}_{b^i}(t)}-{^{b_0^c}\boldsymbol{g}}\right)
	\\\to
	{^{b_0^c}\boldsymbol{g}}&={^{b_0^c}\ddot{\boldsymbol{p}}_{b^i}(t)}-
	{^{b^c_0}_{b^c}\boldsymbol{R}(t)}\cdot{^{b^c}_{b^i}\boldsymbol{R}}
	\cdot{^{b^i}\boldsymbol{a}(t)}
	\\
	{^{b_0^c}_{b^c}\boldsymbol{R}(t)}\cdot{^{b^c}\boldsymbol{a}(t)}&= {^{b_0^c}\ddot{\boldsymbol{p}}_{b^c}(t)}-{^{b_0^c}\boldsymbol{g}}
	\\
	&={^{b_0^c}\ddot{\boldsymbol{p}}_{b^c}(t)}-{^{b_0^c}\ddot{\boldsymbol{p}}_{b^i}(t)}+
	{^{b^c_0}_{b^c}\boldsymbol{R}(t)}\cdot{^{b^c}_{b^i}\boldsymbol{R}}
	\cdot{^{b^i}\boldsymbol{a}(t)}
	\\
	&=-\left(\liehat{{^{b^c_0}_{b^c}\ddot{\boldsymbol{R}}(t)}}
	+\liehat{{^{b^c_0}_{b^c}\dot{\boldsymbol{R}}(t)}}^2\right) \cdot{^{b^c_0}_{b^c}\boldsymbol{R}(t)}\cdot{^{b^c}\boldsymbol{p}_{b^i}}+
	{^{b^c_0}_{b^c}\boldsymbol{R}(t)}\cdot{^{b^c}_{b^i}\boldsymbol{R}}
	\cdot{^{b^i}\boldsymbol{a}(t)}
	\end{aligned}
	\end{equation}
	其中:
	\begin{equation}
	\begin{aligned}
	&{^{b^c_0}\ddot{\boldsymbol{p}}_{b^i}(t)}=
	-\liehat{{^{b^c_0}_{b^c}\boldsymbol{R}(t)}\cdot{^{b^c}\boldsymbol{p}_{b^i}}}\cdot
	{^{b^c_0}_{b^c}\ddot{\boldsymbol{R}}(t)}+{^{b^c_0}\ddot{\boldsymbol{p}}_{b^c}(t)}
	-\liehat{{^{b^c_0}_{b^c}\dot{\boldsymbol{R}}(t)}}\cdot
	\liehat{{^{b^c_0}_{b^c}\boldsymbol{R}(t)}\cdot{^{b^c}\boldsymbol{p}_{b^i}}}\cdot{^{b^c_0}_{b^c}\dot{\boldsymbol{R}}(t)}
	\\
	\to&
	{^{b^c_0}\ddot{\boldsymbol{p}}_{b^c}(t)}-{^{b^c_0}\ddot{\boldsymbol{p}}_{b^i}(t)}=
	\liehat{{^{b^c_0}_{b^c}\boldsymbol{R}(t)}\cdot{^{b^c}\boldsymbol{p}_{b^i}}}\cdot
	{^{b^c_0}_{b^c}\ddot{\boldsymbol{R}}(t)}
	+\liehat{{^{b^c_0}_{b^c}\dot{\boldsymbol{R}}(t)}}\cdot
	\liehat{{^{b^c_0}_{b^c}\boldsymbol{R}(t)}\cdot{^{b^c}\boldsymbol{p}_{b^i}}}\cdot{^{b^c_0}_{b^c}\dot{\boldsymbol{R}}(t)}
	\\
	\to&{^{b^c_0}\ddot{\boldsymbol{p}}_{b^c}(t)}-{^{b^c_0}\ddot{\boldsymbol{p}}_{b^i}(t)}=
	-\liehat{{^{b^c_0}_{b^c}\ddot{\boldsymbol{R}}(t)}}\cdot{^{b^c_0}_{b^c}\boldsymbol{R}(t)}
	\cdot{^{b^c}\boldsymbol{p}_{b^i}}
	-\liehat{{^{b^c_0}_{b^c}\dot{\boldsymbol{R}}(t)}}\cdot
	\liehat{{^{b^c_0}_{b^c}\dot{\boldsymbol{R}}(t)}}\cdot{^{b^c_0}_{b^c}\boldsymbol{R}(t)}
	\cdot{^{b^c}\boldsymbol{p}_{b^i}}
	\\
	\to&{^{b^c_0}\ddot{\boldsymbol{p}}_{b^c}(t)}-{^{b^c_0}\ddot{\boldsymbol{p}}_{b^i}(t)}=
	-\left(\liehat{{^{b^c_0}_{b^c}\ddot{\boldsymbol{R}}(t)}}
	+\liehat{{^{b^c_0}_{b^c}\dot{\boldsymbol{R}}(t)}}^2\right) \cdot{^{b^c_0}_{b^c}\boldsymbol{R}(t)}\cdot{^{b^c}\boldsymbol{p}_{b^i}}
	\end{aligned}
	\end{equation}
	最后有:
	\begin{equation}
	\begin{aligned}
	\int_{\tau\in[t_k,t_{k+1}]}{^{b_0^c}_{b^c}\boldsymbol{R}(\tau)}\cdot{^{b^c}\boldsymbol{a}(\tau)}\cdot d\tau&=
	-\left( \int_{\tau\in[t_k,t_{k+1}]}\left(\liehat{{^{b^c_0}_{b^c}\ddot{\boldsymbol{R}}(t)}}
	+\liehat{{^{b^c_0}_{b^c}\dot{\boldsymbol{R}}(t)}}^2\right) \cdot{^{b^c_0}_{b^c}\boldsymbol{R}(t)}\cdot d\tau\right) \cdot{^{b^c}\boldsymbol{p}_{b^i}}
	\\
	&\;\;\;\;+\int_{\tau\in[t_k,t_{k+1}]}{^{b^c_0}_{b^c}\boldsymbol{R}(t)}\cdot{^{b^c}_{b^i}\boldsymbol{R}}\cdot{^{b^i}\boldsymbol{a}(t)}\cdot d\tau
	\\&=-\mathcal{V}_{t_k,t_{k+1}}^{\prime}\cdot{^{b^c}\boldsymbol{p}_{b^i}}+
	\mathcal{V}_{t_k,t_{k+1}}^{\prime\prime}
	\end{aligned}
	\end{equation}
	
	\section{\normf{速度约束恢复外参}}
	使用最小二乘计算$\{r\}$相对于$\{b_0\}$在$\{r\}$系下的速度向量(即:${^{b}_{r}\boldsymbol{R}^\top}\cdot{^{b_0}_{b}\boldsymbol{R}^\top(\tau)}\cdot{^{b_0}\dot{\boldsymbol{p}}_r(\tau)}$, 其中${^{b_0}\dot{\boldsymbol{p}}_r(\tau)}$为$\{r\}$相对于$\{b_0\}$在$\{b_0\}$系下的速度向量),记为${^{b_0}\dot{\boldsymbol{p}}^r_r(\tau)}$:
	\begin{equation}
	r\cdot v_r=-
	{^{r}\boldsymbol{p}_t^\top(\tau)}\cdot{^{b}_{r}\boldsymbol{R}^\top}\cdot{^{b_0}_{b}\boldsymbol{R}^\top(\tau)}\cdot{^{b_0}\dot{\boldsymbol{p}}_r(\tau)}
	=-{^{r}\boldsymbol{p}_t^\top(\tau)}\cdot{^{b_0}\dot{\boldsymbol{p}}^r_r(\tau)}
	\end{equation}
	样条为(IMU在$\{b_0\}$系下的轨迹):
	\begin{equation}
	{^{b_0}_{b}\boldsymbol{T}(t)}=\left\lbrace {^{b_0}_{b}\boldsymbol{R}(t)}, {^{b_0}\boldsymbol{p}_{b}(t)}\right\rbrace 
	\end{equation}
	有几何变换:
	\begin{equation}
	{^{b_0}\boldsymbol{p}_{r}(t)}={^{b_0}_{b}\boldsymbol{R}(t)}\cdot{^{b}\boldsymbol{p}_{r}}+{^{b_0}\boldsymbol{p}_{b}(t)}
	\end{equation}
	左扰动时间求导:
	\begin{equation}
	{^{b_0}\dot{\boldsymbol{p}}_{r}(t)}=-\liehat{{^{b_0}_{b}\boldsymbol{R}(t)}\cdot{^{b}\boldsymbol{p}_{r}}}\cdot{^{b_0}_{b}\dot{\boldsymbol{R}}(t)}+{^{b_0}\dot{\boldsymbol{p}}_{b}(t)}
	\end{equation}
	所以:
	\begin{equation}
	{^{b_0}\dot{\boldsymbol{p}}^r_r(\tau)}={^{b}_{r}\boldsymbol{R}^\top}\cdot{^{b_0}_{b}\boldsymbol{R}^\top(t)}\cdot{^{b_0}\dot{\boldsymbol{p}}_r(\tau)}={^{b}_{r}\boldsymbol{R}^\top}\cdot{^{b_0}_{b}\boldsymbol{R}^\top(t)}\cdot
	\left(
	{^{b_0}\dot{\boldsymbol{p}}_{b}(t)}-\liehat{{^{b_0}_{b}\boldsymbol{R}(t)}\cdot{^{b}\boldsymbol{p}_{r}}}\cdot{^{b_0}_{b}\dot{\boldsymbol{R}}(t)}
	\right) 
	\end{equation}
	
	\section{\normf{中心约束问题}}
	由于引入了中心约束,所以实际上的批处理优化问题是一个附有限制条件的最小二乘问题。但是,使用Ceres无法求解这样的问题,因为Ceres目前支持的约束只是边界约束(upper and lower bounds constraints)。
	
	对此之前处理的方式是直接在目标函数中增加中心损失残差函数:
	\begin{equation}
	\begin{cases}
	\begin{aligned}
	\boldsymbol{r}_{ctr}(et)&=\sum^{i}_{\mathcal{N}_b}
	{^{b^c}\boldsymbol{p}_{b^i}}
	\\
	\boldsymbol{r}_{ctr}(er)&=\sum^{i}_{\mathcal{N}_b}
	\mathrm{Log}\left( {^{b^c}_{b^i}\boldsymbol{R}}\right)
	\\
	\boldsymbol{r}_{ctr}(tm)&=\sum^{i}_{\mathcal{N}_b}
	{^{b^c}\boldsymbol{\tau}_{b^i}}
	\end{aligned}
	\end{cases}
	\end{equation}
	当然,也可以在求解的时候不加中心约束,直接将某个IMU的外参固定为单位旋转,不进行优化。这样得到的IMU外参姿态就是相对于这个IMU的,即这个IMU就是参考IMU。最后在将所有IMU的外参对齐到中心IMU。
	
	假设现在所有的IMU的外参旋转是相对于参考IMU(不是中心IMU)表示的,若要求解所有的IMU相对于中心IMU的外参旋转,可以先求解中心IMU相对于参考IMU的外参姿态,然后再对齐到中心IMU:
	\begin{algorithm}[h]
		\caption{\normf{计算旋转矩阵的平均矩阵}}
		\LinesNumbered 
		\KwIn{\normf{$\{^{b^r}_{b^i}{\boldsymbol{R}}\}$,其中$\{b^i\}$表示第$i$个IMU,$b^r$表示参考IMU}}
		\KwOut{\normf{$\{^{b^c}_{b^i}{\boldsymbol{R}}\}=\{{^{b^r}_{b^c}\boldsymbol{R}^\top}\cdot{^{b^r}_{b^i}\boldsymbol{R}}\}$,其中$\{b^c\}$表示中心IMU}}
		set ${^{b^r}_{b^c}\boldsymbol{R}}:={^{b^r}_{b^0}\boldsymbol{R}}$, and choose desired tolerance $\epsilon>0$.
	
		\Repeat{$\Vert\boldsymbol{\phi}\Vert>\epsilon$}{
		compute $\boldsymbol{\phi}:=\frac{1}{k}\sum_{i=1}^{k}\mathrm{Log}\left({^{b^r}_{b^c}\boldsymbol{R}^\top}\cdot {^{b^r}_{b^i}\boldsymbol{R}}\right) $
		
		update ${^{b^r}_{b^c}\boldsymbol{R}}:={^{b^r}_{b^c}\boldsymbol{R}}\cdot\mathrm{Exp}\left(\boldsymbol{\phi} \right) $}
		align $\{^{b^c}_{b^i}{\boldsymbol{R}}\}:=\{{^{b^r}_{b^c}\boldsymbol{R}^\top}\cdot{^{b^r}_{b^i}\boldsymbol{R}}\}$
	\end{algorithm}
	
	
	
	
	
	
	
	
	
	
	
	
	
	
\end{document}

